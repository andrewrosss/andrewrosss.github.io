%%%%%%%%%%%%
%The nice Euler/mathpazo font setup is from http://tex.stackexchange.com/questions/59702/suggest-a-nice-font-family-for-my-basic-latex-template-text-and-math/97128#97128
%%%%%%%%%%%%
\documentclass[twoside]{article}
\usepackage[T1]{fontenc}
\usepackage{amsfonts}
\usepackage[tracking]{microtype}
\usepackage[sc,osf]{mathpazo}   % With old-style figures and real smallcaps.
\usepackage{cancel} %for \cancel (strikethrough)
\linespread{1.1}              % Palatino needs a little more leading
% Euler for math and numbers
\usepackage[euler-digits,small]{eulervm}
%\AtBeginDocument{\renewcommand{\hbar}{\hslash}} not sure what this is for

\usepackage{ntheorem} %new theorem
% A new theorem style to make theorem description non-italic, to match the Euler math font.
\makeatletter
\newtheoremstyle{mystyle}
   {\item[\hskip\labelsep \theorem@headerfont ##1\ ##2\theorem@separator]}% 
   {\item[\hskip\labelsep \theorem@headerfont ##1\ ##2\ \textit{(##3)}\theorem@separator]}
\makeatother
\theoremstyle{mystyle}
\theoremheaderfont{\scshape}
\theorembodyfont{\upshape}
\newtheorem{ex}{Exercise}
\newtheorem{exam}{Example}
\newtheorem{fact}{Fact}
\newtheorem{defn}{Definition}
\newtheorem{thm}{Theorem}
\newtheorem{quest}{Question}

%%%%%%%%%
%BEGIN SYMBOL MACROS
%%%%%%%%%
\newcommand{\A}{\mathcal{A}}
\newcommand{\B}{\mathcal{B}}
\newcommand{\C}{\mathcal{C}}
\newcommand{\D}{\mathcal{D}}
\newcommand{\E}{\mathcal{E}}
\newcommand{\F}{\mathcal{F}}
\newcommand{\G}{\mathcal{G}}
\renewcommand{\H}{\mathcal{H}}
\newcommand{\I}{\mathcal{I}}
\newcommand{\K}{\mathcal{K}}
\renewcommand{\L}{\mathcal{L}}
\newcommand{\M}{\mathcal{M}}
\newcommand{\N}{\mathcal{N}}
\let\nothing\O
\renewcommand{\O}{\mathcal{O}}
\renewcommand{\P}{\mathcal{P}}
\newcommand{\Q}{\mathcal{Q}}
\newcommand{\R}{\mathcal{R}}
\renewcommand{\S}{\mathcal{S}}
\newcommand{\U}{\mathcal{U}}
\newcommand{\V}{\mathcal{V}}
\newcommand{\W}{\mathcal{W}}
\newcommand{\Z}{\mathcal{Z}}

\newcommand{\CC}{\mathbb{C}}
\newcommand{\KK}{\mathbb{K}}
\newcommand{\NN}{\mathbb{N}}
\newcommand{\QQ}{\mathbb{Q}}
\newcommand{\RR}{\mathbb{R}}
\newcommand{\ZZ}{\mathbb{Z}}
%%%%%%%%%
%END SYMBOL MACROS
%%%%%%%%%

\usepackage{amsmath}
%\usepackage{hyperref} this was causing trouble - maybe it needs to be above/below a particular "\usepackage" statement...
\usepackage{graphicx} %For \includegraphics
\usepackage{multicol} %For \begin{multicols}{2}etc
\usepackage{enumerate} %For \begin{enumerate}[(a)] etc.
\usepackage{vwcol} %For \begin{vwcol}[widths={0.6,0.4}, sep=.8cm, justify=flush,rule=0pt,indent=1em]
\usepackage{fancyhdr}
\usepackage{lastpage}
\lhead{\large \sc{Mat344}}
\chead{\Large \textbf{\sc{Quiz 1, TUT103}}}
\rhead{Sept 13th, 2017}
\cfoot{}
%\lfoot{Page \thepage \hspace{1pt} of \pageref{LastPage}}
%\rfoot{\texttt{www.math.toronto.edu/rennetad}}

\usepackage[myheadings]{fullpage}

\pagestyle{fancyplain}
\begin{document}

\begin{center}
\section*{\sc{5 Points Available}}
\vspace{0.1in}
\large \underline{\textsc{Instructions}}
\vspace{0.1in}
\normalsize

Please write your \textbf{Name, Student Number,  and Tutorial Number} at the top of this page.\\

\textbf{Remember:} unless you are on the waitlist for the course, you have to write this in your \textbf{registered} tutorial.
\vspace{0.1in}
\end{center}


\begin{center}
%\subsection*{\sc{\underline{Questions}}}
\textit{Make sure to show as many steps of your work as possible, justify as much and annotate any interesting steps or features of your work.  \textbf{Do not just give the final answer.}} \\
%\vspace{0.1in}
%\textbf{Where possible, use - but do not simplify - terms like "$P(n,r)$", "$\binom{n}{k}$",  "$P(n;r_1,...,r_k)$" etc.}
\end{center}

\begin{quest}\mbox{}
Fix some $n\geq 2$.  We select $n+1$ different elements of the set $[2n] = \{1,2,...,2n-1, 2n\}$. Prove  that (no matter our choice of $n$) there will always be two numbers from our selection which are relatively prime (i.e. so that their greatest common divisor is 1.)\\

\textit{Hint: what can we say about consecutive integers, like $k$ \& $k+1$?}
\end{quest}

%There are $n$ pairs of consective integers in $[2n]$.  They are $\{1,2\}, \{3,4\},..., \{2n-1,2n\}$.  By PHP we must pick both numbers from at least one of these pairs.  But notice that $gcd(k,k+1)=1$ always: suppose that $k = ab$ and $k+1 = ac$.  Then $1= k+1 - k = ab-ac = a(b-c)$.  Since $a,b,c$ are integers, $a=1$, showing that any common divisor of them is 1.
\vspace{20pt}

\begin{enumerate}

\item[{\bfseries Solution:}] 

Notice that there are precisely $n$, pairs of consecutive integers when we partition the set $[2n]$ into the sets $\{1,2\}, \{3,4\}, \ldots, \{2n-1, 2n\}$. Let the number of distinct integers we pick ($n+1$) be the pigeons, and let the number of subsets of pairs ($n$) be the holes, by the PHP we have that at least one of the holes must have two pigeons, that is we must have picked two consecutive integers, $i$ and $i + 1$, and since these two numbers are co-prime, we are done. 

\begin{enumerate}

\item[{\bfseries Aside:}] Suppose that $a \mid i$ and $a \mid i + 1$, that is, suppose that $i$ and $i +1$ are not co-prime. We can write $i = ab$ and $i + 1 = ac$, taking the difference $1 = (i + 1) - i = a(c-b)$, since $a, b, c \in \mathbb{Z}$, we have that $a = 1$. Thus any common divisor of $i$ and $i+1$ is 1. 

\end{enumerate}
\end{enumerate}
\end{document}






