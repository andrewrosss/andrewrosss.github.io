%%%%%%%%%%%%
%The nice Euler/mathpazo font setup is from http://tex.stackexchange.com/questions/59702/suggest-a-nice-font-family-for-my-basic-latex-template-text-and-math/97128#97128
%%%%%%%%%%%%
\documentclass[twoside]{article}
\usepackage[T1]{fontenc}
\usepackage{amsfonts}
\usepackage[tracking]{microtype}
\usepackage{amssymb}
\usepackage[sc,osf]{mathpazo}   % With old-style figures and real smallcaps.
\usepackage{cancel} %for \cancel (strikethrough)
\linespread{1.1}              % Palatino needs a little more leading
% Euler for math and numbers
\usepackage[euler-digits,small]{eulervm}
%\AtBeginDocument{\renewcommand{\hbar}{\hslash}} not sure what this is for

\usepackage{ntheorem} %new theorem
% A new theorem style to make theorem description non-italic, to match the Euler math font.
\makeatletter
\newtheoremstyle{mystyle}
   {\item[\hskip\labelsep \theorem@headerfont ##1\ ##2\theorem@separator]}% 
   {\item[\hskip\labelsep \theorem@headerfont ##1\ ##2\ \textit{(##3)}\theorem@separator]}
\makeatother
\theoremstyle{mystyle}
\theoremheaderfont{\scshape}
\theorembodyfont{\upshape}
\newtheorem{ex}{Exercise}
\newtheorem{exam}{Example}
\newtheorem{fact}{Fact}
\newtheorem{defn}{Definition}
\newtheorem{thm}{Theorem}
\newtheorem{quest}{Question}

%%%%%%%%%
%BEGIN SYMBOL MACROS
%%%%%%%%%
\newcommand{\A}{\mathcal{A}}
\newcommand{\B}{\mathcal{B}}
\newcommand{\C}{\mathcal{C}}
\newcommand{\D}{\mathcal{D}}
\newcommand{\E}{\mathcal{E}}
\newcommand{\F}{\mathcal{F}}
\newcommand{\G}{\mathcal{G}}
\renewcommand{\H}{\mathcal{H}}
\newcommand{\I}{\mathcal{I}}
\newcommand{\K}{\mathcal{K}}
\renewcommand{\L}{\mathcal{L}}
\newcommand{\M}{\mathcal{M}}
\newcommand{\N}{\mathcal{N}}
\let\nothing\O
\renewcommand{\O}{\mathcal{O}}
\renewcommand{\P}{\mathcal{P}}
\newcommand{\Q}{\mathcal{Q}}
\newcommand{\R}{\mathcal{R}}
\renewcommand{\S}{\mathcal{S}}
\newcommand{\U}{\mathcal{U}}
\newcommand{\V}{\mathcal{V}}
\newcommand{\W}{\mathcal{W}}
\newcommand{\Z}{\mathcal{Z}}

\newcommand{\CC}{\mathbb{C}}
\newcommand{\KK}{\mathbb{K}}
\newcommand{\NN}{\mathbb{N}}
\newcommand{\QQ}{\mathbb{Q}}
\newcommand{\RR}{\mathbb{R}}
\newcommand{\ZZ}{\mathbb{Z}}
%%%%%%%%%
%END SYMBOL MACROS
%%%%%%%%%

\usepackage{amsmath}
%\usepackage{hyperref} this was causing trouble - maybe it needs to be above/below a particular "\usepackage" statement...
\usepackage{graphicx} %For \includegraphics
\usepackage{multicol} %For \begin{multicols}{2}etc
\usepackage{enumerate} %For \begin{enumerate}[(a)] etc.
\usepackage{vwcol} %For \begin{vwcol}[widths={0.6,0.4}, sep=.8cm, justify=flush,rule=0pt,indent=1em]
\usepackage{fancyhdr}
\usepackage{lastpage}
\lhead{\large \sc{Mat344}}
\chead{\Large \textbf{\sc{Quiz 1, TUT101}}}
\rhead{Sept 13th, 2017}
\cfoot{}
%\lfoot{Page \thepage \hspace{1pt} of \pageref{LastPage}}
%\rfoot{\texttt{www.math.toronto.edu/rennetad}}

\usepackage[myheadings]{fullpage}

\pagestyle{fancyplain}
\begin{document}

\begin{center}
\section*{\sc{5 Points Available}}
\vspace{0.1in}
\large \underline{\textsc{Instructions}}
\vspace{0.1in}
\normalsize

Please write your \textbf{Name and Student Number} at the top of this page.\\

\textbf{Remember:} you have to write quizzes in your \textbf{registered} tutorial.
\vspace{0.1in}
\end{center}


\begin{center}
%\subsection*{\sc{\underline{Questions}}}
\textit{Make sure to show as many steps of your work as possible, justify as much and annotate any interesting steps or features of your work.  \textbf{Do not just give the final answer.}} \\
%\vspace{0.1in}
%\textbf{Where possible, use - but do not simplify - terms like "$P(n,r)$", "$\binom{n}{k}$",  "$P(n;r_1,...,r_k)$" etc.}
\end{center}

\begin{quest} \mbox{}
Prove that there is a positive integer $n$ so that $117^n - 1$ is divisible by 11. (Note: $117 = 3^2\cdot 13$.)\\

\textit{Hint: consider the remainders (mod 11) of the numbers $117-1, 117^2-1, 117^3-1,...,117^{11}-1$.}
\end{quest}
%There are 10 distinct remainders modulo 11.  So, two numbers in the above list have the same remainder modulo 11.  i.e. for some $i<j$ we have $117^j - 1, 117^i-1 \equiv r (mod 11)$.  Taking their difference, we get $(117^j - 1) - (117^i - 1) = 117^i(117^{j-i}-1)$.  But the left-hand side is equal to $r - r=0$ modulo 11.  So, $7\mid 117^i(117^{j-i}-1)$.  But $7 \nmid 117$, so it must be that $7 \mid 117^{j-i}-1$, proving our claim.
\vspace{20pt}

\begin{enumerate}
\item[{\bfseries Solution:}] 

If we have $117^i - 1 \mod 11 \equiv 0$ for some $1 \le i \le 11$ then we are done. 

Suppose for contradiction we have $117^i -1 \mod 11 \in \{1, 2, \ldots, 10\}$, for all $1 \le i \le 11$. In this case, letting the residues of the numbers $117-1, 117^2-1,...,117^{11}-1$ be the pigeons, and letting the numbers $1, 2, \ldots, 10$ be the holes, we have 11 pigeons and 10 holes, therefore by the PHP one of the holes must contain at least two pigeons, that is, we must have that 

\[117^i - 1 \equiv 117^j - 1 \mod 11 \qquad \text{for some $1 \le i < j \le 11$}\]

Taking the difference gives 

\[(117^j - 1) - (117^i - 1) = 117^i(117^{j-i}-1) \equiv 0 \mod 11\]

This implies $11 \mid 117^i(117^{j-i}-1)$. Since $117 = 3^2\cdot 13$, we have that $11 \nmid 117$, thus it must be the case that $11 \mid 117^{j-i}-1$, contradicting the initial assumption that $117^i -1 \mod 11 \in \{1, 2, \ldots, 10\}$, for all $1 \le i \le 11$. 

\end{enumerate}




%It ends up being 117^{10}-1, by the way!
  
\end{document}